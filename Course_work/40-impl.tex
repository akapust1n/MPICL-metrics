\chapter{Технологический раздел}
\section{Выбор  языка программирования}
Для реализации данных алгоритмов был выбран язык  С++. Данный язык был обоснован следующими причинами:
Причины:
\begin{enumerate}
	 \item Компилируемый язык со статической типизацией. 
	 \item Сочетание высокоуровневых и низкоуровневых средств.
	 \item Реализация ООП.
	 \item Наличие удобной стандартной библиотеки шаблонов
	 \end{enumerate}
\section{Выбор вспомогательных библиотек}
Для реализации программы была выбрана библиотека Qt.
\begin{enumerate}
	\item Широкие возможности работы с изображениями, в том числе и попиксельно
	\item Наличии более функциональных аналогов стандартной библиотеки шаблонов в том числе для разнообразных структур данных
\end{enumerate}
Так же были были использованы библиотеки ImageMagick(для конвертации изображения в ограниченную цветовую палитру), OpenMP(многопоточность), OpenGL(работа с шейдерами)
\subsection{Диаграмма классов}
Для реализации различных алгоритмов была разработана следующая структура классов. Был создан абстракный класс дизеринга Dithering с интерфейсом, наследуюмом в дочерних классах. Так же была введена система менджеров: DitherManager, MetricsManager, DataManager и MainManager.


%%% Local Variables:
%%% mode: latex
%%% TeX-master: "rpz"
%%% End:
