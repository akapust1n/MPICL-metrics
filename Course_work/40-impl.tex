\chapter{Технологический раздел}
\section{Выбор  языка программирования}
Для реализации  был выбран язык  С++. Данный язык был обоснован следующими причинами:
Причины:
\begin{enumerate}
	 \item Его поддерживает библиотека PICL
	 \item Компилируемый язык со статической типизацией. 
	 \item Сочетание высокоуровневых и низкоуровневых средств.
	 \item Реализация ООП.
	 \item Наличие удобной стандартной библиотеки шаблонов
	 \end{enumerate}
\section{Выбор вспомогательных библиотек}
Для реализации программы была выбрана библиотека Qt.
\begin{enumerate}
	\item Широкие возможности работы с изображениями, в том числе и попиксельно
	\item Наличии более функциональных аналогов стандартной библиотеки шаблонов в том числе для разнообразных структур данных
\end{enumerate}
Так же были были использованы библиотеки Nodejs,express,jsonwebtoken,mysql-nodejs,mongoose/
\section{Выбор базы данных}
Для хранения файлов трасс была выбрана база данных MySQL
\begin{enumerate}
	\item Быстродействие. Благодаря внутреннему механизму многопоточности быстродействие MySQL весьма высоко.
	\item Лицензия. Раньше лицензирование MySQL было немного запутанным; сейчас эта программа для некоммерческих целей распространяется бесплатно
	\item Переносимость. В настоящее время существуют версии программы для большинства распространенных компьютерных платформ. Это говорит о том, что вам не навязывают определенную операционную систему. Вы сами можете выбрать, с чем работать, например с Linux или Windows, но даже в случае замены ОС вы не потеряете свои данные и вам даже не понадобятся дополнительные инструменты для их переноса.
\end{enumerate}
Для хранения базы данных логинов и паролей была выбрана NoSQL документориентированная база данных MongoDB
\begin{enumerate}
	\item Быстрое извлечение простых структур данных
	
	\item Может хранить неструктурную информацию
	 \item Лицензия. Для некоммерческих целей распространяется бесплатно
\end{enumerate}
\subsection{Диаграмма классов}
Для реализации различных алгоритмов была разработана следующая структура классов. Был создан абстракный класс дизеринга Dithering с интерфейсом, наследуюмом в дочерних классах. Так же была введена система менджеров: DitherManager, MetricsManager, DataManager и MainManager.


%%% Local Variables:
%%% mode: latex
%%% TeX-master: "rpz"
%%% End:
