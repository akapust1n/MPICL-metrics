\Introduction

Вычислительные системы с массовым параллелизмом совершили революцию в мире параллельных вычислений, обеспечив производительность порядка нескольких Терафлоп/с. Необходимость в вычислениях такого объёма возникает в широком круге областей от молекулярной физики до предсказания погоды. На сегодняшний день массивно параллельные системы являются лидерами по критерию цена/производительность.
Однако, за таким прогрессом аппаратного обеспечения не поспевало программное обеспечение. Очевидно, что будут созданы более мощные массивно параллельные системы, и основной проблемой становится разработка программного обеспечения, способного максимально использовать вычислительные ресурсы. Параллельные вычисления связаны с очень сложным и тяжело понимаемым ходом выполнения программ, и недостаточно полное понимание не позволяет полностью использовать вычислительные ресурсы.
Так как производительность – самый важный критерий при использовании параллельных компьютеров, и так как производительность пилотных реализаций параллельных программ как правило далека от ожидаемой, модификация программы с целью повышения производительности является важной частью процесса разработки. Основным фактором, ведущим к большому количеству трудоёмкой работы, требующей высокой квалификации является нехватка средств для анализа и оценки производительности. При отсутствии таких средств, причины низкой производительности выявляются с помощью средств, разработанных для конкретного приложения и метода тыка. Повышение производительности параллельных программ наиболее важная, но трудоёмкая задача, которая требует хорошего понимания задачи. Цель средств контроля и визуализации производительности параллельных программ – помочь разработчику обрести это понимание.
Визуализация производительности заключается в использовании графических образов для представления анализа данных о выполнении программы для лучшего её понимания. Такие средства визуализации были очень полезны и в прошлом, и широко используются в настоящее время для повышения производительности параллельных программ. Несмотря на прогресс в области визуализации научных данных, технологии визуализации для параллельного программирования являются объектом внимания многих разработчиков, так как требуется тем более сложная визуализация, чем более сложными становятся параллельные системы.
